%% LyX 2.3.7 created this file.  For more info, see http://www.lyx.org/.
%% Do not edit unless you really know what you are doing.
\documentclass[10pt,twocolumn,spanish]{article}
\usepackage[T1]{fontenc}
\usepackage[latin9]{inputenc}
\usepackage[a5paper]{geometry}
\geometry{verbose,tmargin=1.5cm,bmargin=1.5cm,lmargin=1.5cm,rmargin=1.5cm,footskip=0.5cm}
\pagestyle{plain}
\setcounter{secnumdepth}{2}
\setcounter{tocdepth}{0}
\setlength{\parskip}{\medskipamount}
\setlength{\parindent}{0pt}
\usepackage{color}
\usepackage{babel}
\addto\shorthandsspanish{\spanishdeactivate{~<>}}

\usepackage{esint}
\usepackage[unicode=true,pdfusetitle,
 bookmarks=true,bookmarksnumbered=false,bookmarksopen=false,
 breaklinks=true,pdfborder={0 0 0},pdfborderstyle={},backref=false,colorlinks=true]
 {hyperref}

\makeatletter

%%%%%%%%%%%%%%%%%%%%%%%%%%%%%% LyX specific LaTeX commands.
\providecommand{\LyX}{\texorpdfstring%
  {L\kern-.1667em\lower.25em\hbox{Y}\kern-.125emX\@}
  {LyX}}

%%%%%%%%%%%%%%%%%%%%%%%%%%%%%% User specified LaTeX commands.
\usepackage{times}
\usepackage{multicol}
\usepackage{fancybox}
\usepackage{calc}
\usepackage{units}
%\usepackage{rotating}
\usepackage{tikz-backgammon}
\renewcommand*{\boardblack}{red}     % \newcommand*{\boardblack}{brown}
\renewcommand*{\boardwhite}{yellow}  % \newcommand*{\boardwhite}{olive!50}
\renewcommand*{\barcolor}{cyan}      % \newcommand*{\barcolor}{brown}
% \renewcommand*{\black}{blue}     % fichas
% \renewcommand*{\white}{green!25}     % fichas

\makeatother

\begin{document}

\subsection*{Mostrando diversos Diagramas de Backgammon}

Se han creado una serie de comandos para representar los diagramas
de Backgammon, el comando indica en la primera parte del como se numeraran
las casillas, si desde el punto de vista del blanco (\emph{\textbackslash whiteboardXX})
o desde el negro (\emph{\textbackslash blackboardXX}). Las dos �ltimas
letras del comando indican donde se colocar� la casa interior del
jugador correspondiente (\emph{U} � \emph{D}: arriba o abajo y \emph{R}
� \emph{L}: derecha o izquierda. Obteni�ndose los comandos:
\begin{itemize}
\item \emph{\textbackslash blackboardDL}
\item \emph{\textbackslash blackboardDR}
\item \emph{\textbackslash blackboardUL}
\item \emph{\textbackslash blackboardUR}
\item \emph{\textbackslash whiteboardDL}
\item \emph{\textbackslash whiteboardDR}
\item \emph{\textbackslash whiteboardUL}
\item \emph{\textbackslash whiteboardUR}
\end{itemize}
Uso comando \emph{\textbackslash blackboardDL} (n�meros casillas
desde el punto de vista del negro) la casa interior del negro abajo
\emph{D} a la izquierda \emph{L}.

{\small{}Nota: En el pre�mbulo del documento he cambiado los colores
de las casillas del negro a rojo }\emph{\small{}\textbackslash renewcommand{*}\{\textbackslash boardblack\}\{red\}}{\small{},
las casillas del blanco a amarillo con }\emph{\small{}\textbackslash renewcommand{*}\{\textbackslash boardwhite\}\{yellow\}}{\small{},
y la barra a color c�an con }\emph{\small{}\textbackslash renewcommand{*}\{\textbackslash barcolor\}\{cyan\}.}{\small\par}

\begin{center}
\resizebox{!}{4cm}{
	\double{neutral}{0}
	% 
	\onbar{white}{1} % ficha blanca en barra
	\onbar{black}{1}
	% las casillas se deben contar desde el punto de vista de cada jugador
	% \whitepoint{n}{0} vacio casilla blancas, para borrar las del anterior tablero
	% no se puede poner \whitepoint{01}{0}  da error el 0 delante del n�mero
	\whitepoint{1}{0}   \whitepoint{2}{2}   \whitepoint{3}{2} 
	\whitepoint{4}{0}   \whitepoint{5}{2}   \whitepoint{6}{2} 
	\whitepoint{7}{2}   \whitepoint{8}{0}   \whitepoint{9}{0} 
	\whitepoint{10}{0}  \whitepoint{11}{0}  \whitepoint{12}{0} 
	\whitepoint{13}{2}  \whitepoint{14}{0}  \whitepoint{15}{0} 
	\whitepoint{16}{0}  \whitepoint{17}{0}  \whitepoint{18}{0} 
	\whitepoint{19}{0}  \whitepoint{20}{0}  \whitepoint{21}{2}
	\whitepoint{22}{0}  \whitepoint{23}{0}  \whitepoint{24}{0} 
	% No escribir las casillas ya ocupadas por el blanco
	\blackpoint{3}{2}   \blackpoint{5}{2}   \blackpoint{6}{3}
	\blackpoint{7}{2}   \blackpoint{8}{2}   
	\blackpoint{13}{2}  \blackpoint{21}{1} 
	%
	\blackroll{23} 
	% 
    \boardcaption{casa interior del negro abajo a la izquierda}
	\blackboardDL % 
}
\end{center}

Uso comando \emph{\textbackslash blackboardDR} (n�meros casillas
desde el punto de vista del negro), cambia la posici�n de la casa
interior del negro al lado derecho \emph{R}.

\begin{center}
\resizebox{!}{4cm}{
	\blackroll{23} 
	% 
    \boardcaption{casa interior del negro abajo a la derecha}
	\blackboardDR  % del blanco
}
\end{center}

Uso comando \emph{\textbackslash whiteboardUL} (n�meros casillas
desde el punto de vista del blanco), con la casa interior del blanco
arriba \emph{U} a la izquierda \emph{L}

\begin{center}
\resizebox{!}{4cm}{
	\double{black}{2}
	% 
	\whiteroll{56}
	%     
	\boardcaption{casa interior del negro abajo a la izquierda}
	\whiteboardUL 
}
\end{center}

Uso comando \emph{\textbackslash whiteboardUR} (n�meros casillas
desde el punto de vista del blanco) con casa interior del blanco arriba
a la derecha

\begin{center}
\resizebox{!}{4cm}{
	\double{black}{2}
	% 
	\whiteroll{56}
	% 
\boardcaption{casa interior del negro abajo a la derecha}
	\whiteboardUR  
}
\end{center}

Uso comando \emph{\textbackslash whiteboardUL} (n�meros casillas
desde el punto de vista del blanco) con casa interior del blanco arriba
\emph{U} a la izquierda \emph{L}

\begin{center}
\resizebox{!}{4cm}{
	\double{black}{2}
	% 
	\blackroll{23} 
	%     
	\boardcaption{casa interior del negro abajo a la izquierda}
	\whiteboardUL 
	}
\end{center}

Uso comando \emph{\textbackslash whiteboardUR} (n�meros casillas
desde el punto de vista del blanco) casa interior del blanco arriba
\emph{U} a la derecha \emph{R}

\begin{center}
\resizebox{!}{4cm}{
	\double{black}{2}
	% 
	\blackroll{23} 
	%\whiteroll{56}
	% 
	\boardcaption{casa interior del negro abajo a la derecha}
	\whiteboardUR
}
\end{center}

Uso comando \emph{\textbackslash blackboardUR} (muestra numeraci�n
casillas desde el punto de vista del negro) con la casa interior del
negro arriba \emph{U} a la derecha \emph{R} 

{\small{}Adem�s he cambiado las casillas negras a color azul con el
comando }\emph{\small{}\textbackslash renewcommand{*}\{\textbackslash boardblack\}\{blue\},
y las casillas blancas a blanco con \textbackslash renewcommand{*}\{\textbackslash boardwhite\}\{white\} }{\small\par}

\renewcommand*{\boardblack}{blue}  
\renewcommand*{\boardwhite}{white}  
\begin{center}
\resizebox{!}{4cm}{
	\double{white}{4}
	% 
	\blackroll{23} 
	% 
	\boardcaption{casa interior del blanco abajo a la derecha}
	\blackboardUR % 
}
\end{center}

Y por fin el diagrama que yo suelo utilizar (y por �l que me met�
en este lio).

{\small{}Adem�s he cambiado el color de las fichas negras a morado
con el comando }\emph{\small{}\textbackslash renewcommand{*}\{\textbackslash black\}\{purple\}}{\small{};
el color de las fichas blancas a naranja con }\emph{\small{}\textbackslash renewcommand{*}\{\textbackslash white\}\{orange\}}{\small{},
las casillas del blanco a amarillo con }\emph{\small{}\textbackslash renewcommand{*}\{\textbackslash boardwhite\}\{yellow\}}{\small{},
y la barra a marr�n con }\emph{\small{}\textbackslash renewcommand{*}\{\textbackslash barcolor\}\{brown\}. }{\small\par}

Uso comando \emph{\textbackslash whiteboardDR} (muestra numeraci�n
casillas desde el punto de vista del blanco) con la casa interior
del blanco abajo \emph{D} a la derecha \emph{R}

\renewcommand*{\boardwhite}{yellow}
\renewcommand*{\barcolor}{brown}  
\renewcommand*{\black}{purple}     % cambio color fichas negras 
\renewcommand*{\white}{orange}     % cambio color fichas blancas
\begin{center}
\resizebox{!}{4cm}{
	\double{white}{4}
	% 
	%\blackroll{23} 
	\whiteroll{56}
	% 
	\boardcaption{casa interior del blanco abajo a la derecha}
	\whiteboardDR  % del blanco
}
\end{center}

Uso comando \emph{\textbackslash whiteboardDL} (muestra numeraci�n
casillas desde el punto de vista del blanco) con la casa interior
del blanco abajo \emph{D} a la izquierda \emph{L}

\renewcommand*{\black}{black}     % cambio color fichas negras 
\renewcommand*{\white}{white}     % cambio color fichas blancas
\begin{center}
\resizebox{!}{4cm}{
	\double{white}{4}
	% 
	%\blackroll{23} 
	\whiteroll{56}
	% 
	\boardcaption{casa interior del blanco abajo a la izquierda}
	\whiteboardDL  % del blanco
}
\end{center}

Uso comando \emph{\textbackslash blackboardUL} (muestra numeraci�n
casillas desde el punto de vista del blanco) con la casa interior
del blanco abajo \emph{D} a la izquierda \emph{L}

\renewcommand*{\black}{black}     % cambio color fichas negras 
\renewcommand*{\barcolor}{cyan}  
\begin{center}
\resizebox{!}{4cm}{
	\double{black}{2}
	% 
	\blackroll{23} 
	%\whiteroll{56}
	% 
	\boardcaption{casa interior del blanco abajo a la izquierda}
	\blackboardUL  % del blanco
}
\end{center}

Despu�s de todas estas pruebas, no se ha roto nada. Y \LyX{} sigue
compilando el PDF correctamente. 

He tenido que modificar los archivos de package:
\begin{itemize}
\item \emph{tikz-backgammon-dice.sty}
\item \emph{tikz-backgammon.sty}
\end{itemize}
El documento lo he creado en \LyX{} usando en el pre�mbulo el comando
\emph{\textbackslash usepackage\{tikz-backgammon\}}, e insertando
sendas cajas de C�digo \TeX{} para elaborar los diagramas.
\end{document}
