%% LyX 2.3.7 created this file.  For more info, see http://www.lyx.org/.
%% Do not edit unless you really know what you are doing.
\documentclass[10pt,twocolumn,spanish]{article}
\usepackage[T1]{fontenc}
\usepackage[latin9]{inputenc}
\usepackage[a5paper]{geometry}
\geometry{verbose,tmargin=1.5cm,bmargin=1.5cm,lmargin=1.5cm,rmargin=1.5cm,footskip=0.5cm}
\pagestyle{plain}
\setcounter{secnumdepth}{2}
\setcounter{tocdepth}{0}
\setlength{\parskip}{\medskipamount}
\setlength{\parindent}{0pt}
\usepackage{color}
\usepackage{babel}
\addto\shorthandsspanish{\spanishdeactivate{~<>}}

\usepackage{esint}
\usepackage[unicode=true,pdfusetitle,
 bookmarks=true,bookmarksnumbered=false,bookmarksopen=false,
 breaklinks=true,pdfborder={0 0 0},pdfborderstyle={},backref=false,colorlinks=true]
 {hyperref}

\makeatletter
%%%%%%%%%%%%%%%%%%%%%%%%%%%%%% User specified LaTeX commands.
\usepackage{times}
\usepackage{multicol}
\usepackage{fancybox}
\usepackage{calc}
\usepackage{units}
%\usepackage{rotating}
\usepackage{tikz-backgammon}
\renewcommand*{\boardblack}{red}     % \newcommand*{\boardblack}{brown}
\renewcommand*{\boardwhite}{yellow}  % \newcommand*{\boardwhite}{olive!50}
\renewcommand*{\barcolor}{cyan}      % \newcommand*{\barcolor}{brown}
% \renewcommand*{\black}{blue}     % fichas
% \renewcommand*{\white}{green!25}     % fichas

\makeatother

\begin{document}

\subsection*{Mostrando diversos Diagramas de Backgammon}

Uso comando \emph{\textbackslash blackboardDL} (n�meros casillas
desde el punto de vista del negro) la casa interior del negro abajo
\emph{D} a la izquierda \emph{L}.

{\footnotesize{}Nota: En el pre�mbulo del documento he cambiado los
colores de las casillas del negro a rojo }\emph{\footnotesize{}\textbackslash renewcommand{*}\{\textbackslash boardblack\}\{red\}}{\footnotesize{},
las casillas del blanco a amarillo con }\emph{\footnotesize{}\textbackslash renewcommand{*}\{\textbackslash boardwhite\}\{yellow\}}{\footnotesize{},
y la barra a color c�an con }\emph{\footnotesize{}\textbackslash renewcommand{*}\{\textbackslash barcolor\}\{cian\}.}{\footnotesize\par}

\begin{center}
\resizebox{!}{4cm}{
	\double{neutral}{0}
	% 
	\onbar{white}{1} % ficha blanca en barra
	\onbar{black}{1}
	% las casillas se deben contar desde el punto de vista de cada jugador
	% \whitepoint{n}{0} vacio casilla blancas, para borrar las del anterior tablero
	% no se puede poner \whitepoint{01}{0}  da error el 0 delante del n�mero
	\whitepoint{1}{0}   \whitepoint{2}{2}   \whitepoint{3}{2} 
	\whitepoint{4}{0}   \whitepoint{5}{2}   \whitepoint{6}{2} 
	\whitepoint{7}{2}   \whitepoint{8}{0}   \whitepoint{9}{0} 
	\whitepoint{10}{0}  \whitepoint{11}{0}  \whitepoint{12}{0} 
	\whitepoint{13}{2}  \whitepoint{14}{0}  \whitepoint{15}{0} 
	\whitepoint{16}{0}  \whitepoint{17}{0}  \whitepoint{18}{0} 
	\whitepoint{19}{0}  \whitepoint{20}{0}  \whitepoint{21}{2}
	\whitepoint{22}{0}  \whitepoint{23}{0}  \whitepoint{24}{0} 
	% No escribir las casillas ya ocupadas por el blanco
	\blackpoint{3}{2}   \blackpoint{5}{2}   \blackpoint{6}{3}
	\blackpoint{7}{2}   \blackpoint{8}{2}   
	\blackpoint{13}{2}  \blackpoint{21}{1} 
	%
	\blackroll{23} 
	% 
    \boardcaption{casa interior del negro abajo a la izquierda}
	\blackboardDL % 
}
\end{center}

Uso comando \emph{\textbackslash blackboardDR} (n�meros casillas
desde el punto de vista del negro), cambia la posici�n de la casa
interior del negro al lado derecho \emph{R}.

\begin{center}
\resizebox{!}{4cm}{
	\blackroll{23} 
	% 
    \boardcaption{casa interior del negro abajo a la derecha}
	\blackboardDR  % del blanco
}
\end{center}

Uso comando \emph{\textbackslash whiteboardUL} (n�meros casillas
desde el punto de vista del blanco), con la casa interior del blanco
arriba \emph{U} a la izquierda \emph{L}

\begin{center}
\resizebox{!}{4cm}{
	\double{black}{2}
	% 
	\whiteroll{56}
	%     
	\boardcaption{casa interior del negro abajo a la izquierda}
	\whiteboardUL 
}
\end{center}

Uso comando \emph{\textbackslash whiteboardUR} (n�meros casillas
desde el punto de vista del blanco) con casa interior del blanco arriba
a la derecha

\begin{center}
\resizebox{!}{4cm}{
	\double{black}{2}
	% 
	\whiteroll{56}
	% 
\boardcaption{casa interior del negro abajo a la derecha}
	\whiteboardUR  
}
\end{center}

Uso comando \emph{\textbackslash whiteboardUL} (n�meros casillas
desde el punto de vista del blanco) con casa interior del blanco arriba
\emph{U} a la izquierda \emph{L}

\begin{center}
\resizebox{!}{4cm}{
	\double{black}{2}
	% 
	\blackroll{23} 
	%     
	\boardcaption{casa interior del negro abajo a la izquierda}
	\whiteboardUL 
	}
\end{center}

Uso comando \emph{\textbackslash whiteboardUR} (n�meros casillas
desde el punto de vista del blanco) casa interior del blanco arriba
\emph{U} a la derecha \emph{R}

\begin{center}
\resizebox{!}{4cm}{
	\double{black}{2}
	% 
	\blackroll{23} 
	%\whiteroll{56}
	% 
	\boardcaption{casa interior del negro abajo a la derecha}
	\whiteboardUR
}
\end{center}

Uso comando \emph{\textbackslash blackboardUR} (muestra numeraci�n
casillas desde el punto de vista del negro) con la casa interior del
negro arriba \emph{U} a la derecha \emph{R} 

{\footnotesize{}Adem�s he cambiado las casillas negras a color azul
con el comando }\emph{\footnotesize{}\textbackslash renewcommand{*}\{\textbackslash boardblack\}\{blue\},
y las casillas blancas a blanco con \textbackslash renewcommand{*}\{\textbackslash boardwhite\}\{white\} }{\footnotesize\par}

\renewcommand*{\boardblack}{blue}  
\renewcommand*{\boardwhite}{white}  
\begin{center}
\resizebox{!}{4cm}{
	\double{white}{4}
	% 
	\blackroll{23} 
	% 
	\boardcaption{casa interior del blanco abajo a la derecha}
	\blackboardUR % 
}
\end{center}

Y por fin el diagrama que yo suelo utilizar (y por �l que me met�
en este lio).

{\footnotesize{}Adem�s he cambiado el color de las fichas negras a
morado con el comando }\emph{\footnotesize{}\textbackslash renewcommand{*}\{\textbackslash black\}\{purple\}}{\footnotesize{};
el color de las fichas blancas a naranja con }\emph{\footnotesize{}\textbackslash renewcommand{*}\{\textbackslash white\}\{orange\}}{\footnotesize{},
las casillas del blanco a amarillo con }\emph{\footnotesize{}\textbackslash renewcommand{*}\{\textbackslash boardwhite\}\{yellow\}}{\footnotesize{},
y la barra a marr�n con }\emph{\footnotesize{}\textbackslash renewcommand{*}\{\textbackslash barcolor\}\{brown\}. }{\footnotesize\par}

Uso comando \emph{\textbackslash whiteboardDR} (muestra numeraci�n
casillas desde el punto de vista del blanco) con la casa interior
del blanco abajo \emph{D} a la derecha \emph{R}

\renewcommand*{\boardwhite}{yellow}
\renewcommand*{\barcolor}{brown}  
\renewcommand*{\black}{purple}     % cambio color fichas negras 
\renewcommand*{\white}{orange}     % cambio color fichas blancas
\begin{center}
\resizebox{!}{4cm}{
	\double{white}{4}
	% 
	%\blackroll{23} 
	\whiteroll{56}
	% 
	\boardcaption{casa interior del blanco abajo a la derecha}
	\whiteboardDR  % del blanco
}
\end{center}

{\small{}Faltar�a implementar la casa interior del blanco abajo a
la izquierda, pero de momento no lo he hecho (no creo necesitarlo).}{\small\par}

Despu�s de todas estas pruebas, no se ha roto nada. Y LyX sigue compilando
el PDF correctamente. 

He tenido que modificar los archivos de package:
\begin{itemize}
\item tikz-backgammon-dice.sty
\item tikz-backgammon.sty
\end{itemize}
El documento lo he creado en LyX usando en el preambulo el comando
\emph{\textbackslash usepackage\{tikz-backgammon\}}, e insertando
sendas cajas de C�digo TeX para elaborar los diagramas.
\end{document}
